\documentclass{article}

\usepackage{fontspec}
\usepackage{polyglossia}
\usepackage{amsmath}
\usepackage{mathtools}

\setmainfont{CMU Serif}

\title{Отчёт: задача о полёте на максимальную дальность}
\author{Serge Kozlukov\\ \texttt{newkozlukov [at] gmail [dot] com}}

\begin{document}
\maketitle
\tableofcontents

\section{Постановка задачи}
Рассмотрим летательный аппарт, не имеющий двигателя, управляемый с помощью
изменения площади несущей поверхности \( S \) и угла атаки \( \alpha \).

Эволюция системы описывается задачей Коши:
\begin{equation}
  \left\{
    \begin{aligned}
      & m\ddot x_1(t) = - R(t)\cos\theta - Y(t)\sin\theta,\\
      & m\ddot x_2(t) = - mg - R(t)\sin\theta - Y(t)\cos\theta,\\
      & x_1(t_0) = x_2(t_0) = 0,\\
      & \dot x_1(t_0) = |V_0|\cos\theta,\\
      & \dot x_2(t_0) = |V_0|\sin\theta,
      \end{aligned}
  \right.
\end{equation}
где
\[ R(t) = \frac12 \rho V(t)^2 S(t) C_x,\]
\[ Y(t) = \frac12 \rho V(t)^2 S(t) C_y.\]

Вводя малый параметр
\[ \varepsilon = \frac{V_0^2}{2mg}, \]
и понижая степень уравнения, получим систему:
\begin{equation}
  \left\{
    \begin{aligned}
      ...
    \end{aligned}
  \right.
\end{equation}
\end{document}